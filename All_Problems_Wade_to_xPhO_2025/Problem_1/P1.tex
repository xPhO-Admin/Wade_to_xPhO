\textbf{Lắc đều trước khi uống}

\textbf{1.} Một cốc nước hình trụ bán kính \(R\), chứa hai loại chất lỏng có khối lượng riêng \(\rho_1\) và \(\rho_2\). Ở trạng thái tĩnh ban đầu, chất lỏng \(\rho_1\) nằm dưới và có độ cao \(h_1\), chất lỏng \(\rho_2\) nằm trên cao \(h_2\). Quay cốc nước này quanh trục đối xứng với vận tốc \(\omega\). Xét trạng thái cốc nước quay ổn định và các chất lỏng không trộn vào nhau. Gia tốc trọng trường là \(g\).

\textbf{a.} Chứng minh rằng mặt ngăn cách giữa hai lớp chất lỏng là mặt Paraboloid.

\textbf{b.} Với các vận tốc góc quay \(\omega\) khác nhau, hình dạng của các mặt nước cũng thay đổi. Hình \ref{fig:Rotating_cup} là hình ảnh phân bố nước trong cốc đối với các trường hợp vận tốc góc \(\omega\) tăng dần từ trái sang phải. Vận tốc góc \(\omega_1\), \(\omega_2\), \(\omega_3\) lần lượt là các vận tốc góc ứng với 3 trường hợp hình \ref{fig:Rotating_cup}b, hình \ref{fig:Rotating_cup}c và \ref{fig:Rotating_cup}d. Xác định điều kiện trị của \(\omega_1\), \(\omega_2\) và \(\omega_3\) theo \(R\), \(h_1\), \(h_2\) và \(g\).

\textbf{c.} Xác định độ cao \(z_{max}\) so với đáy của điểm cao nhất trên mặt nước trong trường hợp hình \ref{fig:Rotating_cup}d.

\begin{figure}[!h]
    \centering
    \includegraphics[width=0.9\linewidth]{Problem_1/Figs_P1/Rotating_cup.pdf}
    \caption{Các trường hợp mặt nước khi tăng dần tốc độ quay của cốc.}
    \label{fig:Rotating_cup}
\end{figure}

\textbf{2.} Thay vì hai chất lỏng ở phần 1, ta đổ vào cốc nước một chất lỏng với độ cao ban đầu \(h_0\). Sau một thời gian, các phần cặn của chất lỏng này lắng xuống đáy cốc, tạo thành sự phân bố khối lượng riêng chất lỏng dạng \(\rho=\rho_0 ( 1 - \alpha z)\). Quay chậm cốc nước này quanh trục đối xứng với vận tốc \(\omega\). Xác định phân bố khối lượng riêng của chất lỏng \(\rho (r, z)\). Xem rằng khi quay cốc, khối lượng riêng của từng phần tử nước không bị thay đổi.

\textit{Lưu ý: Trong bài toán này, ta chọn gốc tính áp suất tương đối bằng không tại một điểm trong không khí.}