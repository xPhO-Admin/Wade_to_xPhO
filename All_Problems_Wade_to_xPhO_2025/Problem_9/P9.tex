\textbf{Hội tụ chùm sóng}

Trong các hệ thống truyền sóng với độ định hướng cao, các nguồn sóng được đặt tại các vị trí xác định cùng nguồn cấp biên độ khác nhau để tạo ra trường giao thoa tập trung vào một hướng. Ở bài toán này, ta sẽ phân tích và thiết kế một số kỹ thuật thiết kế mảng nguồn sóng đơn giản.

\vspace{2mm}

\textbf{1. Hệ số mảng}

Xét một mảng \(N\) nguồn sóng cách đều nhau một khoảng \(d\) trên cùng một đường thẳng với biên độ \(E_0\). Ở vùng rất xa so với mảng nguồn sóng, sóng phát ra của một phần tử thứ \(i\) trong mảng được tính theo thời gian \(t\) và tọa độ \(\mathbf{r}_i\) bằng công thức
\begin{equation}
    E_{i} = E_0 \exp \left[ i \left( \omega t - \mathbf{k}_i \mathbf{r}_i \right) \right],
\end{equation}
với \(\omega\) là tần số góc của sóng, \(\mathbf{k}_i\) là vector số sóng phát ra bởi phần tử thứ \(i\). 

Gọi \(\theta\) là góc xác định hướng truyền sóng với gốc \(theta=0\) ở hướng vuông góc với mảng. Chọn gốc tọa độ \(\mathbf{r}_1 = 0\), 

Khi tính toán giao thoa giữa các sóng, tổng hợp sóng phát ra bởi mảng được tính được dưới dạng
\begin{equation}
    E = \textbf{AF}(\theta) E_0 \exp \left( i \omega t \right)
\end{equation}
với \(\textbf{AF}(\theta)\) được gọi là hệ số mảng.

Tính hệ số mảng \(\textbf{AF}(\theta)\) trong trường hợp này theo \(N\), \(k\) và \(\theta\).

\vspace{2mm}


\textbf{2. Mảng nhị thức}

Tổ hợp chập \(k\) của \(n\) phần tử được định nghĩa là
\begin{equation}
    C_n^k = \dfrac{k! (n-k)!}{n!}.
\end{equation}

Xét một mảng gồm \(N\) nguồn sóng đặt trên một đường thẳng cách đều nhau một khoảng là \(d\). Biên độ sóng do mỗi phần tử nguồn phát sóng tuân theo dãy số nhị thức: \(C_{N-1}^0 E_0\), \(C_{N-1}^1 E_0\), \(C_{N-1}^2 E_0\), \(C_{N-1}^3 E_0\), \ldots, \(C_{N-1}^{N-1} E_0\).

\vspace{2mm}


\textbf{a.} Tính hệ số mảng \(\textbf{AF}(\theta)\) đối với mảng nhị thức bậc \(N\). 

\textbf{b.} Độ rộng nửa búp sóng (HPBW) \footnote{Half Power Beamwidth.} được định nghĩa là góc giữa hai hướng có bình phương biên độ sóng bằng một nửa bình phương biên độ của búp sóng cực đại. Với \(d\) có độ dài bằng một nửa bước sóng, xác định số \(N\) phần tử của mảng để HPBW nhỏ hơn \(30^\circ\).

\vspace{2mm}


\textbf{3. Mảng Dolph-Tschebyscheff}

\textbf{a.} Dãy đa thức Tschebyshev là dãy số tuân theo quy luật
\begin{equation}
    T_m (z) = 2 z T_{m-1} (z) - T_{m-2} (z).
\end{equation}

Chứng minh dãy Tschebyshev có thể được tính theo công thức
\begin{equation}
    T_m (z) = 
    \left\{
    \begin{array}{cc}
        \cos \left[ m \cos^{-1} \left( z \right) \right], & -1<z<1, \\
        \cosh \left[ m \cosh^{-1} \left( z \right) \right], & z<-1, z>1. 
    \end{array}
    \right.
\end{equation}

\textbf{b.} Xét một mảng 8 nguồn sóng đặt trên một đường thẳng cách đều nhau một khoảng là \(d\) với biên độ của các phần tử trong mảng lần lượt là \(a_4 E_0, a_3 E_0, a_2 E_0, a_1 E_0, a_1 E_0, a_2 E_0, a_3 E_0, a_4 E_0\).

Tính các hệ số \(a_1, a_2, a_3, a_4\) sao cho hệ số mảng có dạng
\begin{equation}
    \textbf{AF}(\theta) = \cosh \left\{ 7 \cosh^{-1} \left[ \cos \left( \dfrac{ k d \sin \theta}{2} \right) \right] \right\}.
\end{equation}

