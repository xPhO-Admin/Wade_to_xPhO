Trong các thí nghiệm thường ngày, khi vặn xoắn một vật hình trụ, ta thường nghĩ nó sẽ ngắn lại hoặc giữ nguyên chiều cao (dễ thấy nhất là vặn một lon Pepsi rỗng). Nhưng liệu điều tương tự có xảy ra nếu ta thấy lon Pepsi bằng một vật làm bằng cao su có dạng trụ tròn? Năm 1909, nhà vật lý John Henry Poynting đã chỉ ra rằng một hình trụ cao su khi chịu xoắn sẽ luôn luôn dài ra. Hiện tượng này gọi là hiệu ứng Poynting.

Xét một miếng cao su đặc, không nén, hình trụ, ban đầu có chiều cao là $H$ và bán kính đáy $R_0$. Khi xoắn miếng cao su một góc $\theta$ bằng một thanh cứng, mỏng nhẹ dán chặt vào đáy, chiều cao mới của miếng là $h$. 
\begin{enumerate}
    \item[a.] Tính bán kính sau khi bị xoắn $r_0$ của miếng theo $h$, $H$, $R_0$.
\end{enumerate}
Năm 1943, để mô tả hành vi của một chất rắn mềm có dạng hình hộp khi chịu biến dạng, giáo sư \textit{Leslie Ronald George Treloar} đã đề xuất biểu thức năng lượng đàn hồi của chất rắn ấy dựa trên phân tích nhiệt động học thống kê của mạng lưới polymer liên kết chéo như sau:
\begin{equation}
\tag{*}
U=\frac{\mu_0}{2}\left(\frac{a^2}{A^2}+\frac{b^2}{B^2}+\frac{c^2}{C^2}-3\right) V.
\label{*}
\end{equation}
trong đó $A$, $B$, $C$; $a$, $b$, $c$ lần lượt là kích thước của chất rắn trước và sau khi chịu biến dạng. Thể tích $V=ABC$ là thể tích ban đầu của chất rắn, $\mu_0$ là module trượt.

\begin{figure}[!htb]
    \centering
    \includegraphics[width=0.8\linewidth]{Problem_6/Figs_P6/Deformation.pdf}
    \caption{}
    \label{fig:Deformation}
\end{figure}
\begin{enumerate}
    \item[b.] Biết rằng module trượt của cao su là $\mu$. Hãy tính năng lượng đàn hồi của miếng cao su theo $\mu$, $H$, $h$, $R_0$ và $\theta$.
    \item[c.] Hãy tìm chiều cao sau khi biến dạng $h$ theo $H$, $R_0$, và $\theta$, nếu miếng cao su ban đầu là mỏng và dài.
\end{enumerate}