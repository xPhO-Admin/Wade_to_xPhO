a. Ta xét một lớp vỏ hình trụ ban đầu có bề dày vi phân $d R$, bán kính $R$, biến dạng thành một lớp vỏ cuối cùng có bề dày $d r$, bán kính $r$. Vì cao su là chất bất nén, thể tích vi phân $d V$ của ống trụ mỏng này được bảo toàn trong quá trình biến dạng:
\begin{equation}
\mathrm{d} V=(2 \pi R) H \mathrm{d} R=(2 \pi r) h \mathrm{d} r .
\label{1}
\end{equation}
Tích phân hai vế, ta có:
\begin{equation}
H \int_0^{R} \tilde{R} \mathrm{d} \tilde{R}=h \int_0^{r} \tilde{r} \mathrm{d} \tilde{r} \Rightarrow H R^2=hr^2 .
\label{2}
\end{equation}
Từ phương trình trên, dễ dàng suy ra:
\begin{equation}
r_o=R_0\sqrt{\dfrac{H}{h}} .
\end{equation}
Biểu thức trên cho ta thấy rằng: đứng như dự đoán, nếu miếng cao su dài ra khi bị xoắn, nó phải trở nên mảnh hơn; ngược lại, nếu ngắn đi, nó phải phình to ra.

b. Ta trải phẳng lớp vỏ ở ý a. Kích thước của các lớp này trước và sau khi biến dạng là:
\begin{equation}
\left\{\begin{array}{l}A=2 \pi R \\ B=H \\ C= \mathrm{d} R\end{array} \quad\left\{\begin{array}{l}a=2 \pi r \\ b=\sqrt{h^2+(r \theta)^2} \\ c= \mathrm{d} r .\end{array}\right.\right.
\label{4}
\end{equation}


\begin{figure}[!htb]
    \centering
    \includegraphics[width=0.8\linewidth]{Problem_6/Figs_P6/Fig_B6_sol.pdf}
    \caption{}
    \label{fig:B6_sol}
\end{figure}

Từ phương trình (\ref{*}) và (\ref{4}), năng lượng đàn hồi của lớp vỏ này là:
\begin{equation}
    dU=\frac{\mu}{2} \left[ \frac{r^2}{R^2} +\frac{h^2+(r\theta^2)}{H^2}+\frac{dr^2}{dR^2}   -3\right](2\pi RdR).
\label{5}
\end{equation}
Thay (\ref{1}) và (\ref{2}) vào (\ref{5}):
\begin{equation}
\mathrm{d} U=\mu \pi H\left(2 \frac{H}{h}+\frac{h^2}{H^2}+\frac{R^2 \theta^2}{H h}-3\right) R \mathrm{d} R.
\end{equation}
Năng lượng đàn hồi của miếng cao su là:
\begin{equation}
U=\int_0^{R_0} \mathrm{d} U=\mu \pi H\left[\left(2 \frac{H}{h}+\frac{h^2}{H^2}-3\right) \frac{R_0^2}{2}+\frac{\theta^2}{H h} \frac{R_0^4}{4}\right].
\end{equation}
c. Năng lượng thế toàn phần:
\begin{equation}
E=U(h)-K(h)=U(h).
\end{equation}
Chiều cao $h$ có thể tìm được từ các điều kiện sau:
\begin{equation}
E^{\prime}(h)=\mu \pi R_0^2\left[-\frac{H^2}{h^2}+\frac{h}{H}-\frac{R_0^2 \theta^2}{4 h^2}\right]=0.
\label{9}
\end{equation}
Sau khi xoắn miếng cao su, trạng thái cân bằng của nó là cân bằng bền nên phải thỏa điều kiện:
\begin{equation}
E^{\prime \prime}(h)=\mu \pi R_0^2\left[2 \frac{H}{h^3}+\frac{1}{H}+\frac{R_0^2 \theta^2}{2 h^3}\right] \geq 0.
\label{10}
\end{equation}
Dễ thấy điều kiện ở (\ref{10}) luôn luôn được thỏa mãn. Từ (\ref{9}), chiều cao $h$ của miếng cao su sau khi biến dạng là:
\begin{equation}
h=H \sqrt[3]{1+\dfrac{R_0^2 \theta^2}{4 H^2}}.
\end{equation}
Với $R_0\ll H$ thì:
\begin{equation}
h \simeq H\left(1+\frac{R_0^2 \theta^2}{12 H^2}\right).
\end{equation}

%% Reference %%
\bibliographystyle{plain}
\begin{thebibliography}{}
\bibitem{Zurlo} Zurlo, Giuseppe, et al. "The poynting effect." American Journal of Physics 88.12 (2020): 1036-1040.
\end{thebibliography}
