

\textbf{A.Linear quadrupole trap}  
\vspace{2mm}

\textbf{A.1}

    Trước mắt ta xét điều kiện của điện trường trong không gian bẫy, ta có cường độ điện trường trong bẫy là:
    \begin{equation}
    \notag
        \mathbf{E} = -\mathbf{\nabla} V = -\frac{\partial V}{\partial x}\mathbf{i} - \frac{\partial V}{\partial y}\mathbf{j} = -(a + 2cx)\mathbf{i} - (b + 2dy)\mathbf{j}.
    \end{equation}
    
    Do sự phân bố đối xứng của các ống trụ, cường độ điện trường tổng hợp tại gốc tọa độ bị triệt tiêu bởi 2 ống nằm trên cùng một đường chéo.
    \begin{equation*}
         \mathbf{E}_{(0,0)} = \mathbf{0} \implies \left\{ \begin{array}{l} a = 0 \\ b = 0 \end{array} \right.
    \end{equation*}
 Vì trong bẫy không có phân bố điện tích $\implies \text{div}\mathbf{E} = \rho/\epsilon_0 = 0$
    
     \begin{equation}
     \label{eq:s8_1}
         \frac{\partial E_x}{\partial x} + \frac{\partial E_y}{\partial y} = 0 \implies -2c - 2d = 0 \implies c = -d
     \end{equation}

    Ta xét thêm điều kiện biên của điện thế tại các vị trí đặc biệt 

\begin{equation}
    \left\{
    \begin{array}{l}
        \text{Tại} \  (0, r_0): V_{(0, r_0)} = V_{(0,0)} + d r_0^2 = 0 \\
        \text{Tại} \  (r_0, 0): V_{(r_0, 0)} = V_{(0,0)} + c r_0^2 = V_0.
    \end{array}
    \right.
    \label{eq:s8_2}
\end{equation}

Từ phương trình \ref{eq:s8_1} và \ref{eq:s8_2} , suy ra:

\setlength{\extrarowheight}{9pt}

\begin{equation}
\left\{
\begin{array}{lcr}
      V  & = & \displaystyle\frac{V_0}{2}   \\
      c  &  =   & \displaystyle\frac{V_0}{2r_0^2}  \\
      \text{d}  &  \text{=}   & \text{\(\displaystyle-\frac{V_0}{2r_0^2}\)}
\end{array}
\right.
\label{eq:s8_3}
\end{equation}

Thế vào ta được điện thế V có dạng:
\begin{equation}
    V = \frac{V_0}{2} \left(1 + \frac{x^2 - y^2}{r_0^2}\right).
    \label{eq:s8_4}
\end{equation}

\textbf{A.2}
\vspace{2mm}

Với hàm điện thế đã chứng minh ở trên. Ta dễ dàng suy ra điện trường bằng công thức $\mathbf{E} = -\mathbf{\nabla} V$.
\begin{equation}
\begin{split}
        \mathbf{E} &= -\frac{\partial V}{\partial x}\mathbf{i} - \frac{\partial V}{\partial y}\mathbf{j} \\ &= \left(-\frac{2x}{r_0^2}\frac{V_0}{2}\right)\mathbf{i} - \left(\frac{-2y}{r_0^2}\frac{V_0}{2}\right)\mathbf{j} \\ &= \frac{ V_0}{r_0^2}\left(-x\mathbf{i} + y \mathbf{j} \right).
\end{split}
\label{eq:s8_5}
\end{equation}


Xét một điện tích Q trong bẫy, lực điện tác dụng lên điện tích $Q$ trong bẫy: $\mathbf{F} = Q\mathbf{E}$
\begin{equation}
    \mathbf{F} = \frac{Q V_0}{r_0^2}\left(-x\mathbf{i} + y \mathbf{j} \right).
    \label{eq:s8_6}
\end{equation}

Giả sử $Q V_0 > 0$:
\begin{itemize}
    \item Trên trục $Ox$: $\displaystyle F_x = -\frac{Q V_0}{r_0^2} x$. Lực $F_x$ có dạng lực kéo về (vì $F_x$ và $x$ ngược dấu).
    \item Trên trục $Oy$: $\displaystyle F_y = \frac{Q V_0}{r_0^2} y$. Lực $F_y$ đẩy điện tích ra xa khỏi vị trí cân bằng (vì $F_y$ và $y$ cùng dấu).
\end{itemize}

\textbf{Nhận xét:} Vị trí cân bằng là gốc tọa độ (0,0). Điện tích $Q$ bị đẩy ra xa theo trục $Oy$. Lập luận tương tự với $Q V_0 < 0$, suy ra hàm điện thế tĩnh này không thể giam giữ một hạt điện ổn định.
\vspace{5mm}

\textbf{B. Bẫy Paul hiệu chỉnh}
\vspace{5mm}

\textbf{B.1}
Thay $V_0$ bằng $V(t) = V_{DC} + V_{AC} \cos(\Omega t)$ vào công thức \ref{eq:s8_5}.
\begin{equation}
    \mathbf{E} = \frac{V_{DC} + V_{AC} \cos(\Omega t)}{r_0^2}\left(-x\mathbf{i} + y \mathbf{j} \right)
    \label{eq:s8_7}
\end{equation}

Lực điện tác dụng lên điện tích Q là $\mathbf{F} = Q\mathbf{E}$
\begin{equation}
    \mathbf{F} = \frac{Q\left[V_{DC} + V_{AC} \cos(\Omega t)\right]}{r_0^2}\left(-x\mathbf{i} + y \mathbf{j} \right)
    \label{eq:s8_8}
\end{equation}

Từ định luật II Newton, ta chiếu lên 2 trục thành hai phương trình như sau:

\begin{equation}
\label{eq:s8_9} % Có thể đặt nhãn để tham chiếu
\left\{
\begin{aligned}
    m\ddot{x} &=& -\frac{Q}{r_0^2} \left[V_{DC} + V_{AC} \cos(\Omega t)\right] x \\
    m\ddot{y} &=& \frac{Q}{r_0^2} \left[V_{DC} + V_{AC} \cos(\Omega t)\right] y
\end{aligned}
\right.
\end{equation}

Đổi biến thời gian:

\begin{equation}
\begin{array}{cccccc}
        &   \tau    &   =   &\displaystyle\frac{\Omega t}{2} \\
    \implies    &   \displaystyle\frac{dx}{d\tau}   &   =   &   \displaystyle\frac{dx}{dt}\frac{d\tau}{dt}  &   =   &   \displaystyle\frac{dx}{d\tau}\frac{\Omega}{2}\\
    \implies    &  \displaystyle\frac{d^2 x}{dt^2}&=& \displaystyle\frac{d}{dt} \left(\frac{dx}{d\tau} \frac{\Omega}{2}\right) &=& \displaystyle\frac{d}{d\tau} \left(\frac{dx}{d\tau} \frac{\Omega}{2}\right) \frac{d\tau}{dt} \\

    \implies    &  \displaystyle\frac{d^2 x}{dt^2}&=& \displaystyle\frac{d^2 x}{d\tau^2} \left(\frac{\Omega}{2}\right)^2
\end{array}
\label{eq:s8_10}
\end{equation}
Thế vào phương trình chuyển động theo trục \(x\) (từ 1):
\begin{equation*}
    \begin{array}{cccc}
    &m\ddot{x}  & =&\displaystyle -\frac{Q}{r_0^2}(V_{DC} + V_{AC} \cos(\Omega t)) x \\
    \implies&\displaystyle m \frac{\Omega^2}{4} x''_{\tau}&=& \displaystyle-\frac{Q}{r_0^2}(V_{DC} + V_{AC} \cos(\Omega t)) x. \
    \end{array}
\end{equation*}
Chia cả hai vế cho \( m \Omega^2/4\):

\begin{equation}
    \begin{split}
        &x''_{\tau} = -\left( \frac{4Q V_{DC}}{m\Omega^2 r_0^2} + \frac{4Q V_{AC} \cos(2\tau)}{m\Omega^2 r_0^2} \right) x  \\
        \implies &x''_{\tau} + \left( \frac{4Q V_{DC}}{m\Omega^2 r_0^2} + \frac{4Q V_{AC} \cos(2\tau)}{m\Omega^2 r_0^2} \right) x = 0
    \end{split}
    \label{eq:s8_11}
\end{equation}

Đặt các tham số Mathieu:
\begin{equation*}
    a = \frac{4Q V_{DC}}{m\Omega^2 r_0^2} \quad \text{và} \quad q = \frac{2Q V_{AC}}{m\Omega^2 r_0^2}.
\end{equation*}

Ta rút gọn biểu thức \ref{eq:s8_12} thành

\begin{equation}
    x''_\tau + (a + 2q \cos(2\tau)) x = 0.
    \label{eq:s8_12}
\end{equation}

Tương tự, phương trình chuyển động theo trục \(y\) là:
\begin{equation}
    y''_{\tau} -(a + 2q \cos(2\tau))y = 0.
    \label{eq:s8_13}
\end{equation}

\textbf{B.2}

Ta có: \(V_{DC} = 0 \implies a = 0\). Lúc này, Ion được giam giữ ổn định khi đường $a=0$ trên biểu đồ nằm trong vùng stable (cân bằng trên 2 phương). Từ biểu đồ, ta thấy đường $a=0$ nằm trong vùng ổn định khi \(0 \leq q \leq 0.9\).

\begin{align*}
    \implies& q \leq 0.9 \implies \frac{2Q V_{AC}}{m\Omega^2 r_0^2} \leq 0.9 \\
    \implies& V_{AC} \leq \frac{0.9 \cdot m\Omega^2 r_0^2}{2Q} \\
    \implies& V_{AC} \leq 180.8\, (\text{V})  
\end{align*}
Vậy điện thế hiệu dụng: \( V_{HD} \leq 128\, (\text{V})\).
\vspace{5mm}

\textbf{B.3}
\vspace{2mm}

Sử dụng xấp xỉ đề cho
\begin{equation*}
    x(\tau) = X(\tau) + \epsilon_{x}(\tau), \quad \epsilon_{x}(\tau) \ll X(\tau)
\end{equation*}

Thế vào phương trình Mathieu (\ref{eq:s8_12})
\(x''_{\tau} + (a + 2q \cos(2\tau)) x = 0\):
\begin{equation}
X''(\tau) + \epsilon''_{x} (\tau) + (a + 2q \cos(2\tau)) (X(\tau) + \epsilon_{x} (\tau)) = 0
\label{eq:s8_14}
\end{equation}
Xét chuyển động nhanh (quiver: $\epsilon_{x}$):
\begin{itemize}
    \item[-] Chuyển động này bị chi phối bởi thành phần dao động nhanh (tần số $2\tau$) và xem $X$ là hằng số  \(\left( X(\tau)''\ll \epsilon_{x}(\tau)''\right)\)
    \item[-] Theo đề bài, ta có: $a \ll q$
\end{itemize}
Từ 2 dữ kiện này, ta xấp xỉ: 
\begin{equation}
    \epsilon''_{x} (\tau) \approx -2q X(\tau) \cos(2\tau)
    \label{eq:s8_15}
\end{equation}

Đây là phương trình dao động điều hòa  
\begin{equation}
    \epsilon_{x} (\tau) \approx \frac{q X(\tau)}{2} \cos(2\tau).
    \label{eq:s8_16}
\end{equation}

\noindent
Tương tự cho chuyển động quiver trên trục $y$:
\begin{equation}
\left\{
\begin{array}{l}
    Y''(\tau) + \epsilon''_{y(\tau)} - (a + 2q \cos(2\tau)) (Y(\tau) + \epsilon_{y(\tau)}) = 0 \\ \\
    \epsilon_{y(\tau)} \approx -\dfrac{q \cos(2\tau)}{2} Y(\tau).
\end{array}
\right.
\label{eq:s8_17}
\end{equation}

\vspace{2mm}

\textbf{B.4}
\vspace{2mm}

Ta khảo sát chuyển động trên trục $Ox$ trước.
\vspace{2mm}

Theo phương trình \ref{eq:s8_15}, ta thay biểu thức này vào phương trình \ref{eq:s8_14}, ta được
\begin{equation}
    X''(\tau)  + a  X(\tau)  + (a + 2q \cos(2\tau)) \epsilon_{x} (\tau) = 0.
    \label{eq:s8_18}
\end{equation}

Thay biểu thức \(\epsilon_{x} (\tau) \) vào biểu thức trên:

\begin{equation}
    X''(\tau)  + a  X(\tau)  + \left(\frac{aq}{2} \cos{(2\tau)} + q^2 \cos^2(2\tau)\right) X(\tau) = 0.
    \label{eq:s8_19}
\end{equation}

Khi khảo sát chuyển động Drift, chuyển động quiver lúc này là rất nhanh với tần số góc \(\Omega/2\), do đó phương trình ngay phía trên ta có thể lấy trung bình theo thời gian của chuyển động quiver.


\begin{equation}
    X''(\tau)  + a  X(\tau)  + \left(\frac{aq}{2} \langle \cos{2\tau} \rangle + q^2 \langle \cos^2{2\tau} \rangle \right) X(\tau) = 0.
    \label{eq:s8_20}
\end{equation}


Lại có: 
\(\langle \cos{2\tau} \rangle = 0 \ \text{và} \ \langle \cos^2{2\tau} \rangle = 0.5\). Dẫn tới phương trình chuyển động Drift:
\begin{equation}
    X''(\tau)  + \left(a + \frac{q^2}{2} \right)X(\tau) = 0
    \label{eq:s8_21}
\end{equation}

Phương trình trên là phương trình dao động điều hòa có nghiệm là 
\begin{equation}
    \begin{split}
        &X(\tau)= A\cos{\left(\sqrt{a + \frac{q^2}{2}}\tau + \phi_x\right)} \\
        \implies &X(t)=A\cos{\left(\sqrt{a + \frac{q^2}{2}}\frac{\Omega t}{2} + \phi_x\right)}
    \end{split}
    \label{eq:s8_22}
\end{equation}

\textbf{Nhận xét:} Tần số góc chuyển động Drift rất bé so với chuyển động quiver \(\sqrt{a + \frac{q^2}{2}}\frac{\Omega}{2} \ll \Omega\). Từ đó thỏa mãn giả thiết ở đầu bài.
\vspace{2mm}

Phương trình chuyển động toàn phần trên phương $Ox$:
\begin{equation}
    \begin{split}
        x(t) &=X(t) +\epsilon_{x(t)}=\left(1+\frac{q}{2}\cos{(\Omega t)}\right)X(t) \\
         &=A\left(1+\frac{q}{2}\cos{(\Omega t)}\right)\cos{\left(\sqrt{a + \frac{q^2}{2}}\frac{\Omega t}{2} + \phi_x\right)}.
    \end{split}
    \label{eq:s8_23}
\end{equation}

Tại thời điểm ban đầu: \(t=0\), điện tích đứng yên tại \(x_0\), suy ra điều kiên biên là \(x=x_0\) và \(x'=0\)

\begin{equation}
    \begin{split}
        &\left\{
        \begin{array}{ccc}
        A\left(1+\frac{q}{2}\right)\cos{\phi_x}&=&x_0\\
        -A\left(1+\frac{q}{2}\right)\sin{\phi_x}&=&0\\
        \end{array}
        \right. \\
        \implies&\left\{
        \begin{array}{ccc}
        \sin{\phi_x}&=&0\\
        \cos{\phi_x}&=&1\\
         A&=&\displaystyle \frac{x_0}{\displaystyle \left(1+\frac{q}{2}\right)}
        \end{array}
        \right. \\
    \end{split}
    \label{eq:s8_24}
\end{equation}
Từ đây ta tìm được 
\begin{equation}
    x(t)=\frac{x_0}{\left(1+\frac{q}{2}\right)}\left(1+\frac{q}{2}\cos{(\Omega t)}\right)\cos{\left(\sqrt{a + \frac{q^2}{2}} \ \frac{\Omega t}{2}\right)}.
    \label{eq:s8_25}
\end{equation}

Lặp lại quá trình suy luận và giải phương trình cho chuyển động theo phương $Oy$, ta có được:
\begin{equation}
    y(t)=\frac{y_0}{(1-\frac{q}{2})}\left(1-\frac{q}{2}\cos{(\Omega t)}\right)\cos{\left(\sqrt{-a + \frac{q^2}{2}}\ \frac{\Omega t}{2}\right)}.
    \label{eq:s8_26}
\end{equation}

Vậy phương trình chuyển động của hạt là: 

\begin{equation*}
\left\{
    \begin{array}{l}
        x(t)=\frac{x_0}{\left(1+\frac{q}{2}\right)}\left(1+\frac{q}{2}\cos{(\Omega t)}\right)\cos{\left(\sqrt{a + \frac{q^2}{2}} \ \frac{\Omega t}{2}\right)} \\ 
        x(t)=\frac{y_0}{(1-\frac{q}{2})}\left(1-\frac{q}{2}\cos{(\Omega t)}\right)\cos{\left(\sqrt{-a + \frac{q^2}{2}}\ \frac{\Omega t}{2}\right)}.
    \end{array}
\right.
\end{equation*}


*Lưu ý: Vì \( a\ll q \ll 1\), cho nên kết quả biên độ dao động Drift \((A_x,A_y)=(x_0,y_0)\) và tần số góc dao động Drift \(\displaystyle\frac{\sqrt2}{4}q\Omega\) Vẫn được cho điểm tối đa.

\textbf{B.5}
\vspace{2mm}

Ta biết chuyển động Drift \(\left(X, Y \right)\) là hai dao động điều hoà. Ta sẽ có hệ phương trình vi phân.
\begin{equation}
\left\{
\begin{array}{ccc}
 mX(t)''&=& -m\left(\displaystyle a+\frac{q^2}{2}\right)X\\
 mY(t)''&=& -m\left(\displaystyle-a+\frac{q^2}{2}\right)Y.
\end{array}
\right.
\label{eq:s8_27}
\end{equation}

Vì chuyển động Drift được coi như chịu tác dụng của lực Ponderomotive, ta có:

\begin{equation}
\left\{
\begin{array}{ccccc}
F_x &=& mX(t)''&=& -m\left(\displaystyle a+\frac{q^2}{2}\right)X\\
F_y &=& mY(t)''&=& -m\left(\displaystyle-a+\frac{q^2}{2}\right)Y.
\end{array}
\right.
\label{eq:s8_28}
\end{equation}

Từ đó, ta có được phương trình lực Ponderomotive như sau: 

\[
\mathbf{F_{pond}}=-m\left(a+\frac{q^2}{2}\right)X\mathbf{i}-m\left(-a+\frac{q^2}{2}\right)Y\mathbf{j}
\]

Thế năng do lực Ponderomotive gây ra là:
\begin{equation}
\begin{split}
    U_{pond}&=-\int \mathbf{F_{pond}}.\mathbf{\mathrm{d}r}=-\int_0^x \left(F_{pond}\right)_x\mathrm{d}X -\int_0^y \left(F_{pond}\right)_y\mathrm{d}Y \\
    &= -\int_0^x  -m\left(a+\frac{q^2}{2}\right)X\mathrm{d}X-\int_0^y -m\left(-a+\frac{q^2}{2}\right)Y\mathrm{d}Y \\
    &= \frac{1}{2}m\left(a+\frac{q^2}{2}\right)X^2 +\frac{1}{2}m\left(-a+\frac{q^2}{2}\right)Y^2 
\end{split}
\end{equation}


*Nhận xét: Hàm thế năng lúc này có dạng giống thế năng đàn hồi do lò xo gây ra trên cả 2 phương \(x\) và \(y\), khiến cho điện tích Q bị kéo về vị trí cân bằng (0,0) trên cả 2 phương, dẫn tới bẫy sẽ nhốt điện tích một cách hiệu quả hơn.

%% Reference %%
\bibliographystyle{plain}
\begin{thebibliography}{}
\bibitem{Major}
Major, Fouad G., Viorica N. Gheorghe, and Achim Weiss.
\textit{Charged Particle Traps: Physics and Techniques of Charged Particle Field Confinement}. Vol. 37.
Berlin: Springer, 2005.

\end{thebibliography}